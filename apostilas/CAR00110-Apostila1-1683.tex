% Options for packages loaded elsewhere
\PassOptionsToPackage{unicode}{hyperref}
\PassOptionsToPackage{hyphens}{url}
%
\documentclass[
]{article}
\usepackage{amsmath,amssymb}
\usepackage{iftex}
\ifPDFTeX
  \usepackage[T1]{fontenc}
  \usepackage[utf8]{inputenc}
  \usepackage{textcomp} % provide euro and other symbols
\else % if luatex or xetex
  \usepackage{unicode-math} % this also loads fontspec
  \defaultfontfeatures{Scale=MatchLowercase}
  \defaultfontfeatures[\rmfamily]{Ligatures=TeX,Scale=1}
\fi
\usepackage{lmodern}
\ifPDFTeX\else
  % xetex/luatex font selection
\fi
% Use upquote if available, for straight quotes in verbatim environments
\IfFileExists{upquote.sty}{\usepackage{upquote}}{}
\IfFileExists{microtype.sty}{% use microtype if available
  \usepackage[]{microtype}
  \UseMicrotypeSet[protrusion]{basicmath} % disable protrusion for tt fonts
}{}
\makeatletter
\@ifundefined{KOMAClassName}{% if non-KOMA class
  \IfFileExists{parskip.sty}{%
    \usepackage{parskip}
  }{% else
    \setlength{\parindent}{0pt}
    \setlength{\parskip}{6pt plus 2pt minus 1pt}}
}{% if KOMA class
  \KOMAoptions{parskip=half}}
\makeatother
\usepackage{xcolor}
\usepackage[margin=1in]{geometry}
\usepackage{color}
\usepackage{fancyvrb}
\newcommand{\VerbBar}{|}
\newcommand{\VERB}{\Verb[commandchars=\\\{\}]}
\DefineVerbatimEnvironment{Highlighting}{Verbatim}{commandchars=\\\{\}}
% Add ',fontsize=\small' for more characters per line
\usepackage{framed}
\definecolor{shadecolor}{RGB}{248,248,248}
\newenvironment{Shaded}{\begin{snugshade}}{\end{snugshade}}
\newcommand{\AlertTok}[1]{\textcolor[rgb]{0.94,0.16,0.16}{#1}}
\newcommand{\AnnotationTok}[1]{\textcolor[rgb]{0.56,0.35,0.01}{\textbf{\textit{#1}}}}
\newcommand{\AttributeTok}[1]{\textcolor[rgb]{0.13,0.29,0.53}{#1}}
\newcommand{\BaseNTok}[1]{\textcolor[rgb]{0.00,0.00,0.81}{#1}}
\newcommand{\BuiltInTok}[1]{#1}
\newcommand{\CharTok}[1]{\textcolor[rgb]{0.31,0.60,0.02}{#1}}
\newcommand{\CommentTok}[1]{\textcolor[rgb]{0.56,0.35,0.01}{\textit{#1}}}
\newcommand{\CommentVarTok}[1]{\textcolor[rgb]{0.56,0.35,0.01}{\textbf{\textit{#1}}}}
\newcommand{\ConstantTok}[1]{\textcolor[rgb]{0.56,0.35,0.01}{#1}}
\newcommand{\ControlFlowTok}[1]{\textcolor[rgb]{0.13,0.29,0.53}{\textbf{#1}}}
\newcommand{\DataTypeTok}[1]{\textcolor[rgb]{0.13,0.29,0.53}{#1}}
\newcommand{\DecValTok}[1]{\textcolor[rgb]{0.00,0.00,0.81}{#1}}
\newcommand{\DocumentationTok}[1]{\textcolor[rgb]{0.56,0.35,0.01}{\textbf{\textit{#1}}}}
\newcommand{\ErrorTok}[1]{\textcolor[rgb]{0.64,0.00,0.00}{\textbf{#1}}}
\newcommand{\ExtensionTok}[1]{#1}
\newcommand{\FloatTok}[1]{\textcolor[rgb]{0.00,0.00,0.81}{#1}}
\newcommand{\FunctionTok}[1]{\textcolor[rgb]{0.13,0.29,0.53}{\textbf{#1}}}
\newcommand{\ImportTok}[1]{#1}
\newcommand{\InformationTok}[1]{\textcolor[rgb]{0.56,0.35,0.01}{\textbf{\textit{#1}}}}
\newcommand{\KeywordTok}[1]{\textcolor[rgb]{0.13,0.29,0.53}{\textbf{#1}}}
\newcommand{\NormalTok}[1]{#1}
\newcommand{\OperatorTok}[1]{\textcolor[rgb]{0.81,0.36,0.00}{\textbf{#1}}}
\newcommand{\OtherTok}[1]{\textcolor[rgb]{0.56,0.35,0.01}{#1}}
\newcommand{\PreprocessorTok}[1]{\textcolor[rgb]{0.56,0.35,0.01}{\textit{#1}}}
\newcommand{\RegionMarkerTok}[1]{#1}
\newcommand{\SpecialCharTok}[1]{\textcolor[rgb]{0.81,0.36,0.00}{\textbf{#1}}}
\newcommand{\SpecialStringTok}[1]{\textcolor[rgb]{0.31,0.60,0.02}{#1}}
\newcommand{\StringTok}[1]{\textcolor[rgb]{0.31,0.60,0.02}{#1}}
\newcommand{\VariableTok}[1]{\textcolor[rgb]{0.00,0.00,0.00}{#1}}
\newcommand{\VerbatimStringTok}[1]{\textcolor[rgb]{0.31,0.60,0.02}{#1}}
\newcommand{\WarningTok}[1]{\textcolor[rgb]{0.56,0.35,0.01}{\textbf{\textit{#1}}}}
\usepackage{graphicx}
\makeatletter
\def\maxwidth{\ifdim\Gin@nat@width>\linewidth\linewidth\else\Gin@nat@width\fi}
\def\maxheight{\ifdim\Gin@nat@height>\textheight\textheight\else\Gin@nat@height\fi}
\makeatother
% Scale images if necessary, so that they will not overflow the page
% margins by default, and it is still possible to overwrite the defaults
% using explicit options in \includegraphics[width, height, ...]{}
\setkeys{Gin}{width=\maxwidth,height=\maxheight,keepaspectratio}
% Set default figure placement to htbp
\makeatletter
\def\fps@figure{htbp}
\makeatother
\setlength{\emergencystretch}{3em} % prevent overfull lines
\providecommand{\tightlist}{%
  \setlength{\itemsep}{0pt}\setlength{\parskip}{0pt}}
\setcounter{secnumdepth}{-\maxdimen} % remove section numbering
\ifLuaTeX
  \usepackage{selnolig}  % disable illegal ligatures
\fi
\IfFileExists{bookmark.sty}{\usepackage{bookmark}}{\usepackage{hyperref}}
\IfFileExists{xurl.sty}{\usepackage{xurl}}{} % add URL line breaks if available
\urlstyle{same}
\hypersetup{
  pdftitle={CAR00110},
  pdfauthor={Prof.~Lucas Helal, MMSc, PhD},
  hidelinks,
  pdfcreator={LaTeX via pandoc}}

\title{CAR00110}
\usepackage{etoolbox}
\makeatletter
\providecommand{\subtitle}[1]{% add subtitle to \maketitle
  \apptocmd{\@title}{\par {\large #1 \par}}{}{}
}
\makeatother
\subtitle{~Aprendizado em Linguagem R: Estatística Descritiva I\\
\hspace*{0.333em}Apostila I}
\author{Prof.~Lucas Helal, MMSc, PhD}
\date{2023-08-16}

\begin{document}
\maketitle

\hypertarget{exercuxedcios}{%
\subsection{1.4 Exercícios}\label{exercuxedcios}}

Como é costumeiro, começaremos aprendendo a linguagem R como qualquer
outra linguagem de programação: por meio de um programa \emph{``Hello
World''}. Com isso, iremos aprender que você pode programar em R tanto
no \textbf{prompt de comando} - ou \textbf{console}, quanto em um
\textbf{script} ou mesmo em documento dinâmico (que veremos mais a
frente).

\hypertarget{o-prompt-de-comandoconsole}{%
\subsubsection{1.1. O prompt de
comando/console}\label{o-prompt-de-comandoconsole}}

Uma vez com o \emph{setup} do seu R Studio (ou qualquer outro ambiente
de desenvolvimento) configurado, a tela de \texttt{Console} aparecerá
como uma das abas dos seus quadrantes que ficam à esquerda da tela -
esta é a forma nativa carregada pelo R Studio, mas você pode
reconfigurar a posição como quiser. De forma fácil, você simplesmente
digita o comando direto no \texttt{prompt}/\texttt{console}. \(\\\)
\textbf{Você verá algo como}: \(\\\)

\hypertarget{descricao}{%
\label{descricao}}%
\begin{Shaded}
\begin{Highlighting}[]
\NormalTok{R version 4.3.0 (2023{-}04{-}21) {-}{-} Already Tomorrow}
\NormalTok{Copyright (C) 2023 The R Foundation for Statistical Computing}
\NormalTok{Platform: x86\_64{-}apple{-}darwin20 (64{-}bit)}

\NormalTok{R is free software and comes with ABSOLUTELY NO WARRANTY.}
\NormalTok{You are welcome to redistribute it under certain conditions.}
\NormalTok{Type \textquotesingle{}license()\textquotesingle{} or \textquotesingle{}licence()\textquotesingle{} for distribution details.}

\NormalTok{Natural language support but running in an English locale.}

\NormalTok{R is a collaborative project with many contributors.}
\NormalTok{Type \textquotesingle{}contributors()\textquotesingle{} for more information and}
\NormalTok{\textquotesingle{}citation()\textquotesingle{} on how to cite R or R packages in publications.}

\NormalTok{Type \textquotesingle{}demo()\textquotesingle{} for some demos, \textquotesingle{}help()\textquotesingle{} for on{-}line help, or}
\NormalTok{\textquotesingle{}help.start()\textquotesingle{} for an HTML browser interface to help.}
\NormalTok{Type \textquotesingle{}q()\textquotesingle{} to quit R.}

\NormalTok{\textgreater{}}
\end{Highlighting}
\end{Shaded}

\(\\\) \(\\\) Se você digitar e apertar \texttt{enter} diretamente:
\(\\\) \(\\\)

\hypertarget{descricao}{%
\label{descricao}}%
\begin{Shaded}
\begin{Highlighting}[]
\NormalTok{\textgreater{} 2 + 2 }
\NormalTok{[1] 4}
\end{Highlighting}
\end{Shaded}

\(\\\) \(\\\) Ou\ldots{} \(\\\) \(\\\)

\begin{Shaded}
\begin{Highlighting}[]
\NormalTok{\textgreater{} myString \textless{}{-} "Hello, World!"}
\NormalTok{\textgreater{} print(myString)}

\NormalTok{[1] "Hello, World!"}
\end{Highlighting}
\end{Shaded}

\(\\\) \(\\\) Onde {[}\(n\){]} sempre sinalizará um \textbf{output de
resultado}. \(\\\) \(\\\) Na sentença \#1, você atribuiu a palavra
(\texttt{string}) \textbf{Hello World} ao objeto \texttt{myString}.
\(\\\) \(\\\) Na sentença \#2, você aplicou uma função que indica à IDE,
em linguagem R, que você está dando o comando para que a IDE/Linguagem R
leia o que está contido dentro do objeto previamente criado - a função
\texttt{print} por meio do comando \texttt{print()}.

\hypertarget{o-r-script}{%
\subsubsection{1.2. O R Script}\label{o-r-script}}

Nem sempre você quererá rodar códigos e não tê-los em mãos para utilizar
outra vez, ou mesmo otimiza-lo. Por conta disso, é muito comum que em
qualquer linguagem de programação ou pacote estatístico, que o código
seja escrito em um documento na IDE que se está usando ou mesmo
utilizando editores de texto simples (bloco de notas, html, etc.).

No R, a gente possui como primeiro recurso o R Script, que é um
documento em que você escreve todas as linhas de código antes de
visualizar o resultado dos códigos.

\begin{itemize}
\item
  Caso você queira escrever algo que não faça parte do código, utilize o
  símbolo \texttt{\#} antes de começar a escrever. Assim o R Studio não
  entenderá o que está escrito como código
\item
  Números e palavras são entendidos diferentemente em linguagem R. Para
  números, não há a necessidade de nenhum recurso adicional. Para
  palavras, você deve utilizar as aspas \texttt{"nonono"} - dessa forma,
  o R Studio não compreenderá a palavra como comentário, e sim como
  parte do código que você está escrevendo
\item
  A linguagem R possui a característica marcante de ser orientada à
  objetos. Ou seja, nós digitamos uma parte de código e transformamos em
  um objeto, que pode ser re-transformado em outro objeto à medida que
  você for acumulando informações. Por exemplo: \texttt{1+2} deve ser
  \textbf{idealmente } escrito como \texttt{a\ \textless{}-\ 1+2}. Para
  pedir o resultado, em uma nova linha de comando, escreva a letra
  \texttt{a} (no caso desse exemplo) e rode o comanmdo.
\end{itemize}

\hypertarget{tipos-de-variuxe1veisdados-em-linguagem-r}{%
\subsubsection{1.3. Tipos de variáveis/dados em linguagem
R}\label{tipos-de-variuxe1veisdados-em-linguagem-r}}

Em R, temos uma variedade de tipos de dados/variáveis ou classes de
objetos. Aqui listo os mais frequentes para uso, assumindo que você irá
aprender ainda muito mais à medida que continua estudando linguagem R.
\(\\\) Para fins didáticos, dividirei os tipos de variáveis de acordo
com a classificação estatística. \(\\\) \(\\\) \textbf{PS: aqui vai uma
pequena lista com símbolos mais utilizados em Estatística (com os
comandos para documento Markdown).} \(\\\) \(\\\) \(x\) - x/Observação.
\textbf{Comando} --\textgreater{} \texttt{\$x\$} \(\\\) \(\\\) \(x_1\) -
O primeiro valor observado. \textbf{Comando} --\textgreater{}
\texttt{\$x\_1\$} \(\\\) \(\\\) \(\overline x\) - A média amostral.
\textbf{Comando} --\textgreater{}
\texttt{\$\textbackslash{}overline\ x\$} \(\\\) \(\\\) \(\mu\) - A média
populacional (parâmetro). \textbf{Comando} --\textgreater{}
\texttt{\$\textbackslash{}mu\$} \(\\\) \(\\\) \(\hat p\) - A proporção
em uma amostra. \textbf{Comando} --\textgreater{}
\texttt{\$\textbackslash{}hat\ p\$} \(\\\) \(\\\) \(\hat P\) - A
proporção em uma população (parâmetro). \textbf{Comando}
--\textgreater{} \texttt{\$\textbackslash{}hat\ p\$} \(\\\) \(\\\)
\(n,N\) - Tamanhos amostrais e populacionais, respectivamente.
\textbf{Comando} --\textgreater{} \texttt{\$n,N\$} \(\\\) \(\\\) \(s\) -
Desvio padrão amostral. \textbf{Comando} --\textgreater{} \texttt{\$s\$}
\(\\\) \(\\\) \(s^2\) - Variância amostral. \textbf{Comando}
--\textgreater{} \texttt{\$s\^{}2\$} \(\\\) \(\\\) \(\sigma\) - Desvio
padrão populacional (parâmetro). \textbf{Comando} --\textgreater{}
\texttt{\$\textbackslash{}sigma\$} \(\\\) \(\\\) \(\sigma^2\) -
Variância populacional (parâmetro). \textbf{Comando} --\textgreater{}
\texttt{\$\textbackslash{}sigma\^{}2\$} \(\\\) \(\\\) \(|\ y\ |\) - O
valor absoluto de uma variável (módulo). \textbf{Comando}
--\textgreater{}
\texttt{\$\textbar{}\textbackslash{}\ y\textbackslash{}\ \textbar{}\$}
\(\\\) \(\\\) \(k\) - Denota \emph{múltiplas aparições} para qualquer
variável. \textbf{Comando} --\textgreater{} \texttt{\$k\$} \(\\\) \(\\\)
\(\\\) \(\\\) \textbf{Variáveis Numéricas ou Quantitativas} \(\\\)
\(\\\) Em Matemática e Estatística, uma variável numérica ou
quantitativa é aquela que pode ser definida como \textbf{contínua} ou
\textbf{discreta}, de forma que as variáveis contínuas podem ser
definidas como: \(\\\) \(\\\) \emph{Contínuas} \(\\\) \(\\\) São aquelas
variáveis que pertecem ao conjunto dos números reais e que podem ser
expandidas por números decimais de forma indefinida. Em outras palavras,
variáveis contínuas \textbf{medem} quantidades em uma dimensão, como a
\textbf{distância} entre dois pontos; a \textbf{temperatura} ambiente.
\(\\\) \(\\\) \[
\mathbb{R}\ \ \exists\ \Longleftrightarrow\  [A]\ onde:\ \forall\ \ a_n\ \therefore\ há\ pelo\ menos\ um\ -\infty\ \leq\ a\ \leq + \infty \tag{1.0}
\] \(\\\) \(\\\) Ou: \(\\\) \(\\\) \[
\mathbb{R}\ é\ o\ conjunto\ dos\ números\ reais\ existe\ se,\ e\ somente\ se,\ existir\ um\ conjunto\ qualquer\ \mathbb{A}\ ... \tag{1.1}
\] \(\\\) \(\\\) \emph{Complete a sentença\ldots{}} \(\\\) \(\\\) E,
daqui, é importante saber que: \(\\\) \(\\\) \[
X\ \  é\ \ contínua\ \ e\ \in\ \mathbb{R} \tag{1.3}
\] \(\\\) \(\\\) \emph{Variáveis contínuas pertencem ao conjunto dos
números reais, mas nem todos os elementos do conjunto dos números reais
são variáveis contínuas.} \(\\\) \(\\\) Em R, variáveis contínuas são
chamadas de \texttt{numéricas}; ou, mais precisamente, \texttt{numeric}.
\(\\\) \(\\\) \[
A=\{10.7,\ 92, -43.223,\ \pi,\ \sqrt27,\ \ \sin \theta\} \\ \tag{1.4}
\] \(\\\) \(\\\) Exemplos de fenônemos que podem ser explicados por
variáveis contínuas: \(\\\) - O peso corporal

\begin{itemize}
\item
  A altura humana
\item
  Os valores de troponina ultrassensível
\item
  O diâmetro de uma artéria coronária descendente anterior
\item
  A renda familiar\ldots{} \(\\\) \(\\\) \(\\\) \(\\\) \(\\\) \(\\\)
  \(\\\) \(\\\) \(\\\) \(\\\) \(\\\) \(\\\) \(\\\) \(\\\) \(\\\) \(\\\)
  \(\\\) \(\\\) \(\\\) \(\\\) \(\\\) \(\\\) \emph{Contínuas: Construindo
  uma distribuição normal em Linguagem R} \(\\\) \(\\\)
\end{itemize}

\includegraphics{CAR00110-Apostila1-1683_files/figure-latex/unnamed-chunk-3-1.pdf}

\includegraphics{CAR00110-Apostila1-1683_files/figure-latex/unnamed-chunk-4-1.pdf}
\(\\\) \(\\\) \(\\\) \(\\\) \textbf{Quer construir VOCÊ uma distribuição
normal em linguagem R?} \(\\\) \(\\\) \(\\\)

\hypertarget{exercuxedcios-1}{%
\subsubsection{1.4 Exercícios}\label{exercuxedcios-1}}

\textbf{Em todos os exercícios você deverá utilizar boas práticas em
programação (ex: comentar o código, identificar corretamente os objetos,
não omitir etapas que parecem óbvias etc.).}

\(\\\) \(\\\)

\textbf{Estes deverão ser realizados em R Markdown.}

\begin{quote}
\url{https://livro.curso-r.com/9-2-r-markdown.html}
\end{quote}

\(\\\) \(\\\) \emph{1.4.1. Operações matemáticas} \(\\\) \(\\\) - a)
Você deve realizar 8 operações matemáticas básicas, sendo 2 para soma, 2
para subtração, 2 para multiplicação e 2 para divisão. Para cada tipo de
operação, uma delas deve ser feita com números e a outra com objetos

\begin{itemize}
\item
  \begin{enumerate}
  \def\labelenumi{\alph{enumi})}
  \setcounter{enumi}{1}
  \tightlist
  \item
    Escreva uma operação em modo função de primeiro grau (ex: f(x) = ax
    + b). Escreva-a no .Rmd em linguagem LaTeX (lista completa de
    símbolos:
    \url{https://linorg.usp.br/CTAN/info/symbols/comprehensive/symbols-a4.pdf})
    e identifique o que cada termo da equação quer dizer em Estatística,
    com um exemplo voltado à saúde.
  \end{enumerate}
\item
  \begin{enumerate}
  \def\labelenumi{\alph{enumi})}
  \setcounter{enumi}{2}
  \tightlist
  \item
    Crie um vetor, armazenado em um objeto qualquer, de palavras simples
    e compostas
  \end{enumerate}
\item
  \begin{enumerate}
  \def\labelenumi{\alph{enumi})}
  \setcounter{enumi}{3}
  \tightlist
  \item
    Crie um vetor, armazenando em um objeto qualquer, que misture
    palavras e números contínuos \(\\\) \(\\\) \emph{1.4.2. Linguagem R
    e Estatística Descritiva} \(\\\) \(\\\)
  \end{enumerate}
\item
  \begin{enumerate}
  \def\labelenumi{\alph{enumi})}
  \setcounter{enumi}{4}
  \tightlist
  \item
    Realize a plotagem de um gráfico hipotético da distribuição normal.
    Abaixo há instruções detalhadas da estrutura do código. \(\\\)
    \(\\\)
  \end{enumerate}
\end{itemize}

\begin{Shaded}
\begin{Highlighting}[]

\NormalTok{{-}  Defina os atributos do banco de dados fictício que você irá criar. Você vai precisar:}

\NormalTok{  {-}  Conjunto de dados observados (eixo X)}
\NormalTok{  {-}  Distribuição de probabilidades deste conjunto, com base na Normal teórica}
\NormalTok{  {-}  Utilização correta da função que gerará a distribuição desejada.}
  
  
\NormalTok{{-}  Aqui vai um esqueleto pra você:}


\NormalTok{  {-}  X: \textless{}nome\_do\_objeto\_1\textgreater{} \textless{}{-} seq(min,max, length=n) {-} Este comando indica }
\NormalTok{  que você definirá, para os valores de X, um banco fictício, no formato sequência }
\NormalTok{  de valores. Você deve colocar um valor mínimo, um valor máximo, e no atributo length, }
\NormalTok{  indicar a quantidade de números no vetor}
  
  
\NormalTok{  {-}  Y: \textless{}nome\_do\_objeto\_2\textgreater{} \textless{}{-} dnorm(\textless{}objeto\_que\_representa\_X\textgreater{}) {-} Com este comando, }
\NormalTok{  você cria um novo banco, que é referente às probabilidades de cada valor X ser observado. }
\NormalTok{  Como falado em aula, um gráfico de distribuição de observações possui em seu eixo X }
\NormalTok{  os valores possíveis e observados, e no eixo Y, a probabilidade acumulada deste }
\NormalTok{  valor ocorrer. O comando indicado significa: você criará uma distribuição normal }
\NormalTok{  dos valores de $X$, utilizando a função dnorm, e indicando, dentro dela, o objeto}
\NormalTok{  previamente criado que representa o vetor $X$ }
  
  
\NormalTok{  {-} \textless{}nome\_do\_objeto\_3\textgreater{} \textless{}{-} plot(x, y, xlab="legenda do eixo X", ylab="legenda no eixo Y", }
\NormalTok{  main="Título" ) {-} função plot gera o gráfico com base nos atributos colocados}
  
  
\NormalTok{  {-}  Quarto passo {-}{-}\textgreater{} agora é com você! :)}


\NormalTok{\#  Caso tenha dúvidas ou algum comando não rode, utilize, no console, }
\NormalTok{a função help("\textless{}nome\_da\_função\textgreater{}") ou digite ?\textless{}nome\_da\_função\textgreater{}.}


\NormalTok{\#  Por vezes o que pode travar um código é o uso de letras maiúsculas; acentos; }
\NormalTok{a ausência de parênteses para fechar cada bloco de código; a ausência de aspas }
\NormalTok{nas palavras; espaços não{-}solicitados.}
\end{Highlighting}
\end{Shaded}

\(\\\) \(\\\) - f) Indique, por favor, a estimativa da média, da
mediana, da probabilidade máxima encontrada e do desvio-padrão amostral.
Neste momento não é necessário utilizar nenhuma função para retornar
tais valores - tente identificar visualmente e aponte no documento os
locais de referência utilizados \(\\\) \(\\\) \emph{1.4.3. Pré-exposição
a conteúdo futuro} \(\\\) \(\\\) - g) Aqui a ideia é que você tente por
si mesmo/a identificar conceitos fundamentais que aprenderemos na
próxima aula. Use e abuse da literatura de apoio, assim como da minha
disponibilidade e do nosso fórum via Discord. :) \(\\\) \(\\\) - Traga o
conceito formal e o conceito prático de uma variável discreta, com
referências

\begin{itemize}
\item
  Tente escrever a notação formal da mesma tal qual foi feito neste
  documento. Use a lógica :)
\item
  Identifique o nome computacional deste tipo de variável. É universal
\item
  Crie um vetor que contenha somente variáveis discretas, armazene-o em
  um objeto e leia-o
\item
  Peça formalmente em linguagem R para que o R Studio teste se trata-se
  de uma variável discreta mesmo ou não (TRUE ou FALSE)
\item
  Identifique o conceito de \texttt{array} em linguagem R
\item
  Identifique o ceonceito de \texttt{fator} em linguagem R
\item
  Identifique o conceito de \texttt{data\ frame} em linguagem R
\item
  Identifique o conceito de \texttt{pipe} em linguagem R, traga sua
  simbologia e exemplifique como uma sintaxe com e sem \texttt{pipe} é
  escrita
\item
  Identifique o conceito de \texttt{packages} em R e tente listar os 10
  pacotes essenciais para o estudo da Estatística e da Epidemiologia que
  não vêm nativos quando fazemos o download da linguagem R. Com uma
  breve busca na internet, em fóruns, é fácil de encontrar.
\item
  Demonstre como se \texttt{instala} um pacote e como se \texttt{lê} um
  pacote utilizando o ambiente R Studio.
\end{itemize}

\begin{center}\rule{0.5\linewidth}{0.5pt}\end{center}

\end{document}
